\documentclass[a4paper,10pt]{article}
\usepackage[a4paper]{geometry}
\geometry{hscale=0.8,vscale=0.8,centering}
\usepackage[utf8]{inputenc}
\usepackage[T1]{fontenc}
\usepackage[french]{babel} % Exposant


\usepackage{enumerate} % Listes
\usepackage{amsmath} % Matrices
\usepackage{graphicx}
\usepackage{amssymb}
\usepackage{ulem}
\usepackage{color}

\usepackage{listings} % Lecture du code
\usepackage{hyperref} % Hyperlien


% Mise en page spéciale fancyhdr pour les en-têtes
\usepackage{fancyhdr}
\pagestyle{fancy}

\renewcommand{\headrulewidth}{0pt}
\fancyhead[C]{} % Rien en haut de page au milieu
\fancyhead[L]{\leftmark} % Nom du chapitre actuel en haut de page à gauche
\fancyhead[R]{\thepage} % Numéro de page en haut de page à droite

\renewcommand{\footrulewidth}{0pt}
\fancyfoot[C]{} % Rien en bas de page au milieu
\fancyfoot[L]{Open source pour ajout ou modification: https://github.com/Rodriguevb/GEOGF106-Societe\_et\_environnement-Resume/} % Numéro de page en bas de page à gauche
\fancyfoot[R]{\thepage} % Numéro de page en bas de page à droite
\setlength{\headheight}{12.1638pt}


\author{Rodrigue \textsc{Van Brande}} % Auteur
\date{\today} % Date de compilation du pdf

\pdfinfo{
    /Author   (Rodrigue VAN BRANDE)
    /Creator  (https://github.com/Rodriguevb/GEOGF106-Societe_et_environnement-Resume/)
}

%Création d'un subsub...section (avec \paragraph{} \subparagraph{})
\setcounter{secnumdepth}{5}
\setcounter{tocdepth}{5}


%\newcommand{\fonction}[nb de parametre]{définition de la commande}
\newcommand{\guillemets}[1]{\og #1 \fg} % citation entre guillemets
\newcommand{\gras}[1]{\textbf{#1}} %met en gras
\newcommand{\italique}[1]{\textit{#1}} %met en italique

% Titre du PDF
\title{GEOG-F-106 - Société et environnement\\Jean-Michel \textsc{Decroly}\\Frank \textsc{Pattyn}\\Résumé du cours}

\pdfinfo{
/Title(GEOG-F-106 - Société et environnement - Résumé du cours)
}

%Profondeur pour la table des matières dans les titres
\setcounter{tocdepth}{5}

\begin{document}
    \maketitle       % Titre
    \newpage         % Saut de page
    \tableofcontents % table des matières / Besoin d'une double compilation
    \newpage         % Saut de page

    \section{Autres aides}

\section{Autres aides}

\section{Autres aides}

\input{3.autres_aides/1.tableau_du_formulaire/text}
\input{3.autres_aides/2.densite_et_repartition/text}
\input{3.autres_aides/3.distributions/text}
\section{Autres aides}

\input{3.autres_aides/1.tableau_du_formulaire/text}
\input{3.autres_aides/2.densite_et_repartition/text}
\input{3.autres_aides/3.distributions/text}
\section{Autres aides}

\input{3.autres_aides/1.tableau_du_formulaire/text}
\input{3.autres_aides/2.densite_et_repartition/text}
\input{3.autres_aides/3.distributions/text}
\section{Autres aides}

\section{Autres aides}

\input{3.autres_aides/1.tableau_du_formulaire/text}
\input{3.autres_aides/2.densite_et_repartition/text}
\input{3.autres_aides/3.distributions/text}
\section{Autres aides}

\input{3.autres_aides/1.tableau_du_formulaire/text}
\input{3.autres_aides/2.densite_et_repartition/text}
\input{3.autres_aides/3.distributions/text}
\section{Autres aides}

\input{3.autres_aides/1.tableau_du_formulaire/text}
\input{3.autres_aides/2.densite_et_repartition/text}
\input{3.autres_aides/3.distributions/text}
\section{Autres aides}

\section{Autres aides}

\input{3.autres_aides/1.tableau_du_formulaire/text}
\input{3.autres_aides/2.densite_et_repartition/text}
\input{3.autres_aides/3.distributions/text}
\section{Autres aides}

\input{3.autres_aides/1.tableau_du_formulaire/text}
\input{3.autres_aides/2.densite_et_repartition/text}
\input{3.autres_aides/3.distributions/text}
\section{Autres aides}

\input{3.autres_aides/1.tableau_du_formulaire/text}
\input{3.autres_aides/2.densite_et_repartition/text}
\input{3.autres_aides/3.distributions/text}
    
\end{document}