\newpage
\subsection{Variables aléatoires particulières}


\subsubsection{Variable binomiale$\mathcal{B}(n,p)$}
\subsubsection{Variable de Poisson $\mathcal{P}_\lambda$}
\subsubsection{Variable exponentielle négative}
\subsubsection{Variable Normale $\mathcal{N}(\mu,\sigma)$}




\paragraph{Propriété d'opposé}\label{propriete-normale-oppose}
Soit $\bar{X_1}$ et $\bar{X_2}$ deux variables normales. Si $\bar{X_2}$ est l'opposé de $\bar{X_1}$ ( $\bar{X_2}$ = -$\bar{X_1}$ ) alors
\begin{center}
$\boxed{\bar{X}_1 \sim N \left( \mu_1, \sigma \right)$ et $\bar{X}_2 \sim N \left( -\mu_2, \sigma \right)}$
\end{center}




\paragraph{Propriété d'addition}\label{propriete-normale-addition}
Soit $N(\mu_1,\sigma_1)$ et $N(\mu_2,\sigma_2)$ deux variables normales indépendantes et $\psi_1(t)$ et $\psi_2(t)$ leurs fonctions génératrices.
\begin{center}
$\begin{array}{LL}
\psi(t) &= \psi_1(t).\psi_2(t)\\
        &= e^{\mu_1+t+\sigma_1^2t^2/2}.e^{\mu_2+t+\sigma_2^2t^2\frac{1}{2}}\\
        &= e^{(\mu_1+\mu_2)t+\frac{1}{2}(\sigma_1^2+\sigma_2^2)t^2}\\
\end{array}$
\end{center}
On obtient bien une fonction génératrice d'une normale de paramètres $(\mu_1+\mu_2)$ et $\sqrt{\sigma_1^2+\sigma_2^2}$.
$$\boxed{N(\mu_1,\sigma_1) + N(\mu_2,\sigma_2) = N\left(\mu_1+\mu_2,\sqrt{\sigma_1^2+\sigma_2^2}\right)}$$







\subsubsection{Variable Khi$^2$}
\subsubsection{Variable Student $t_n$}
\label{propriete-student}
\subsubsection{Variable Snedecor $\mathcal{F}_{(m,n)}$}


